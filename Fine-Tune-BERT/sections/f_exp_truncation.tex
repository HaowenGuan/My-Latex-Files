\section{Analysis}
\label{sec:exp}

In this section, we conduct experiments that help us understand why Pre-trained Distillation is successful and how to attribute credit to its constituent operations.

\subsection{There Are No Shortcuts: Why Full Pre-training Is Necessary}

As later elaborated in Section \ref{sec:related}, earlier efforts to leverage pre-training in the context of compact models simply feed pre-trained (possibly contextual) input representations into randomly-initialized students \citep{distillation-reading-comprehension, distillation-classification,tang2019distilling}. Concurrent work initializes shallow-and-wide students from the bottom layers of their deeper pre-trained counterparts \citep{truncated-pretraining, patient_kd}. The experiments below indicate these strategies are suboptimal, and that LM pre-training is necessary in order to unlock the full student potential. 

\paragraph{Is it enough to pre-train word embeddings?}
\emph{No.} In order to prove that pre-training Transformer layers is important, we compare two flavors of \recipename\footnote{Note that, for both flavors of PD, none of the student parameters are frozen; the word embeddings do get updated during distillation.}:
PD with pre-trained word embeddings and PD with pre-trained word embeddings \emph{and} Transformer layers.  We produce word-piece embeddings by pre-training one-layer Transformers for each embedding size. We then discard the single Transformer layer and keep the  embeddings to initialize our students.

For MNLI (Figure \ref{fig:3b-pretraining-ablation}), less than 24\% of the gains PD brings over distillation can be attributed to the pre-trained word embeddings (for \berttiny, this drops even lower, to 5\%). The rest of the benefits come from additionally pre-training the Transformer layers.

\paragraph{Is it worse to truncate deep pre-trained models?}
\emph{Yes, especially for shallow students.} Given that pre-training is an expensive process, an exhaustive search over model sizes in the pursuit of the one that meets a certain performance threshold can be impractical. Instead of pre-training all (number of layers, embedding size) combinations of students, one way of short-cutting the process is to pre-train a single deep (e.g.~12-layer) student for each embedding size, then truncate it at various heights. Figure \ref{fig:3b-pretraining-ablation} shows that this can be detrimental especially to shallow architectures; \berttiny loses more than 73\% of the pre-training gains over distillation. As expected, losses fade away as the number of layers increases.

\paragraph{{What is the best student for a fixed parameter size budget?}}
\emph{As a rule of thumb, prioritize depth over width, especially with pre-trained students.} Figure~\ref{fig:all_students} presents a comparison between 24 student model architectures on SST-2, demonstrating how well different students utilize model capacity. They are sorted first by the hidden size, then by the number of layers. This roughly corresponds to a monotonic increase in the number of parameters, with a few exceptions for the largest students. The quality of randomly initialized students (i.e. basic training and distillation) is closely correlated with the number of parameters. With pre-training (i.e. PD and PF), we observe two intuitive findings: (1) pre-trained models are much more effective at using more parameters, and (2) pre-trained models are particularly efficient at utilizing depth, as indicated by the sharp drops in performance when moving to wider but shallower models.

This is yet another argument against initialization via truncation: for instance, truncating the bottom two layers of BERT\textsubscript{BASE} would lead to a suboptimal distribution of parameters: the 2L/768H model (39.2m parameters) is dramatically worse than e.g. 6L/512H (35.4m parameters). 

\begin{figure}[t]
    \vspace{-10pt}
    \small
    \begin{minipage}[t]{0.48\textwidth}
    \centering
    \scalebox{0.9}{
    \begin{tikzpicture}[trim left=0]
    \begin{axis}[
            % If you change this to true, also s/bert_sizes_v2.dat/bert_sizes.dat
            enlargelimits=true,
            xticklabels from table={bert_sizes.dat}{Label},
            xticklabel style={align=center,font=\small},
            %every tick label/.append style={font=\footnotesize},
            %scale only axis,
            %scaled y ticks=false,
            title style={font=\small},
            legend style={font=\footnotesize, align=center},
            legend columns=1,
            legend pos=south east,
            legend cell align={left},
            title=MNLI,
            ymin=58,
        ]
        \addbertplot{\deftmarker}{\deftcolor}{solid}{data/accuracies_5points/D-mnli-snli-qqp_P-bw.csv};
        \addbertplot{*}{gray}{solid}{data/accuracies_5points/D-nli_P-bw-truncated.csv};
        \addbertplot{\wpmarker}{\wpcolor}{solid}{data/accuracies_5points/D-mnli-snli-qqp_P-wp.csv};
        \addbertplot{\kdmarker}{\kdcolor}{solid}{data/accuracies_5points/D-mnli-snli-qqp_P-rnd.csv};
    
        \addlegendentry{PD (MLM pre-training)}
        \addlegendentry{PD (from 12-layer model)}
        \addlegendentry{PD (word embeddings)}
        \addlegendentry{Distillation}
    \end{axis}
    \end{tikzpicture}
    } % end scalebox
    \caption{\small \textbf{Pre-training outperforms truncation.} Students initialized via LM pre-training (green) outperform those initialized from the bottom layers of 12-layer pre-trained models (gray). When only word embeddings are pre-trained (red), performance is degraded even further.}
    \label{fig:3b-pretraining-ablation}
    \end{minipage}
    %
    \hspace{10pt}
    %
    \begin{minipage}[t]{0.48\textwidth}
    \centering
    \scalebox{0.9}{
    \begin{tikzpicture}[trim left=0]
        \pgfaccuracynoscale
        \begin{axis}[legend columns=2,
                     legend pos=north east,
                     transpose legend=false,
                     title=\sst,
                     title style={font=\small},
                     legend style={font=\small},
                     ymin=79.9999,
                     ymax=100]
            \addbertplot{\deftmarker}{\deftcolor}{solid}{data/accuracies/D-movies_P-bw.csv};
            \addlegendentry{PD}
            \addbertplot{none}{red}{dashed}{data/accuracies/sst2_teacher.csv};
            \addlegendentry{Teacher}
            \addbertplot{\pretrainmarker}{\pretraincolor}{solid}{data/accuracies/F-sst2_P-bw.csv};
            \addlegendentry{PF}
            \addbertplot{\vtmarker}{\vtcolor}{dashed}{data/accuracies/F-sst2_P-rnd.csv};
            \addlegendentry{Basic training}
            \addbertplot{\kdmarker}{\kdcolor}{solid}{data/accuracies/D-movies_P-rnd.csv};
            \addlegendentry{Distillation}
        \end{axis}
    \end{tikzpicture}
    } % end \scalebox
    \caption{\small \textbf{Depth outweighs width} when models are pre-trained (PD and PF), as emphasized by the sharp drops in the plot. For instance, the 6L/512H (35.4m parameters) model outperforms the 2L/768H model (39.2m parameters). Randomly initialized models take poor advantage of extra parameters.}
    \vspace{-20pt}
    \label{fig:all_students}
    \end{minipage}
\end{figure}