\subsection{Better Together: The Compound Effect of Pre-training and Distillation}

\begin{wrapfigure}{t}{0.5\textwidth}
    \centering
    \vspace{-15pt}
    \scalebox{0.75}{
    \begin{tikzpicture}
    \begin{axis}[
            % If you change this to true, also s/bert_sizes_v2.dat/bert_sizes.dat
            enlargelimits=true,
            xticklabels from table={bert_sizes.dat}{Label},
            xticklabel style={align=center, font=\small},
            % every tick label/.append style={font=\small},
            %scale only axis,
            %scaled y ticks=false,
            title style={font=\small},
            legend style={font=\footnotesize, align=center},
            legend columns=1,
            legend pos=south east,
            legend cell align={left},
            title=MNLI,
        ]
        
        \addbertplot{square}{\deftcolor}{dashed}{data/accuracies_5points/D-nli_P-nli.csv};
        \addlegendentry{PD (\DLM = \DT = \nlistar)}
        \addbertplot{diamond}{\pretraincolor}{dashed}{data/accuracies_5points/F-mnli_P-mnli-snli-qqp-pairs.csv};
        \addlegendentry{PF (\DLM = \nlistar)}
        
        \addbertplot{\kdmarker}{\kdcolor}{solid}{data/accuracies_5points/D-mnli-snli-qqp_P-rnd.csv};
        \addlegendentry{Distillation (\DT = \nlistar)}
    \end{axis}
    \end{tikzpicture}
    } % end scalebox
    \caption{\small {\bf Pre-training complements distillation.} PD outperforms the baselines even when we pre-train and distill on the same dataset (\DLM = \DT = \nlistar).}
    \label{fig:3a-pretraining-ablation}
    \vspace{-10pt}
\end{wrapfigure}

We investigate the interaction between pre-training and distillation by applying  them sequentially on the \emph{same} data. We compare the following two algorithms: \PtFt with \DLM = $X$ and \recipename with \DLM = \DT = $X$. Any additional gains that the latter brings over the former must be attributed to distillation, providing evidence that the compound effect still exists.

For MNLI, we set \DLM = \DT = \nlistar and continue the experiment above by taking the students pre-trained on \DLM = \nlistar and distilling them on \DT = \nlistar. As shown in Figure \ref{fig:3a-pretraining-ablation}, PD is better than PF by 2.2\% on average over all student sizes. Note that even when pre-training and then distilling on the \emph{same data}, PD outperforms the two training strategies applied in isolation. The two methods are thus learning different linguistic aspects, both useful for the end task.

% We'll add this for camera ready. I didn't have time to finish all experiments.
% \subsection{Less Is More: Encouraging the Community to Build Smaller Models}
% In order to encourage the community to build more compact models, we establish a baseline for \berttiny (2L/128H)
% in Table \ref{tab:tiny_results}. We provide results for \PtFt (PF), as well as Pre-trained Distillation (PD and PDF$^+$) using BERT\textsubscript{BASE} (12L/768H) as a teacher. Note that this is in contrast to our experimental section, where the teacher is BERT\textsubscript{LARGE}.
%\begin{table}[]
    \vspace{-10pt}
    \centering
    %\resizebox{\textwidth}{!}{
    \begin{tabular}{l|l|g|c|c|c|c|c|c}
         \toprule
         & Model & Meta & SST-2 & MRPC & QQP & MNLI & QNLI & RTE \\
         & 
            & Score 
            & \footnotesize (acc)
            & \footnotesize (f1/acc)
            & \footnotesize (f1/acc)
            & \footnotesize (acc m/mm)
            & \footnotesize (acc)
            & \footnotesize (acc) \\
         \midrule
         \parbox[t]{3mm}{\multirow{3}{*}{\rotatebox[origin=c]{90}{test}}}
         & PF &
            TODO &      % Score
            TODO &      % SST-2
            TODO &      % MRPC
            TODO &      % QQP
            TODO &      % MNLI
            TODO &      % QNLI
            TODO \\     % RTE
         & PD &
            74.1 &              % Score
            83.5 &              % SST-2
            80.8/71.2 &         % MRPC
            63.0/84.0 &         % QQP
            72.9/72.6 &         % MNLI
            79.8 &              % QNLI
            59.1 \\             % RTE
         & PD$^+$F &
            TODO &      % Score
            TODO &      % SST-2
            TODO &      % MRPC
            TODO &      % QQP
            TODO &      % MNLI
            TODO &      % QNLI
            TODO \\     % RTE
         \midrule
         \parbox[t]{3mm}{\multirow{3}{*}{\rotatebox[origin=c]{90}{dev}}}
         & PF &
            TODO &              % Score
            81.8 &              % SST-2
            81.5/71.3 &         % MRPC
            81.7/85.6 &         % QQP
            TODO &      % MNLI
            78.5 &              % QNLI
            59.1 \\             % RTE
         & PD &
            75.9 &              % Score
            81.7 &              % SST-2
            82.4/73.0 &         % MRPC
            82.3/86.2 &         % QQP
            72.7/73.2 &         % MNLI
            79.2 &              % QNLI
            59.1 \\             % RTE
         & PD$^+$F &
            TODO &      % Score
            TODO &      % SST-2
            TODO &      % MRPC
            TODO &      % QQP
            TODO &      % MNLI
            TODO &      % QNLI
            TODO \\     % RTE
         \bottomrule
    \end{tabular}
    %} % end \resizebox
    % A tabular is "fragile command" in a "moving argument". See https://texfaq.org/FAQ-extrabrace.
    \caption[sanitization argument]{
        \textbf{Baselines for future work}:\vspace{2mm}
        \small
        \begin{tabular}{ll}
            PF: & \PtFt \\
            PD: & Pre-trained Distillation on \DT$=$\DL \\
            PD$^+$F: & Pre-trained Distillation on \DT$>$\DL, fine-tuned on \DL \\
        \end{tabular}
        All students are 2L/128H BERT models, trained by 12L/768H BERT teachers. For PD and PD$^+$F, softmax temperature was set to 100. Hyperparameter sweeps include: batch size (32, 64, 128), learning rate (2e-05, 5e-05) and training epochs (3, 4, 10). Dev results are averaged over 5 runs with different seeds for the best hyperparameter setting. Test results are evaluated on the GLUE server, using the model that performed best on dev. The meta score is computed on 6 tasks only, and it is \underline{not} directly comparable to the GLUE leaderboard.
    }
    \label{tab:tiny_results}
    \vspace{-10pt}
\end{table}