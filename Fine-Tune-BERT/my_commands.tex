\newcommand{\eat}[1]{}
%\DeclareMathOperator*{\E}{\mathbb{E}}
%\aclfinalcopy % Uncomment this line for the final submission

%\setlength\titlebox{5cm}
% You can expand the titlebox if you need extra space
% to show all the authors. Please do not make the titlebox
% smaller than 5cm (the original size); we will check this
% in the camera-ready version and ask you to change it back.

% Teacher
\newcommand\bertlarge{BERT\textsubscript{LARGE}\xspace}
% We decided to call students "Transformer" instead of BERT. But I'm too lazy to update the name of the macro.
\newcommand\bertbase{Transformer\textsubscript{BASE}\xspace}
\newcommand\bertsmall{Transformer\textsubscript{SMALL}\xspace}
\newcommand\bertmini{Transformer\textsubscript{MINI}\xspace}
\newcommand\berttiny{Transformer\textsubscript{TINY}\xspace}
\newcommand\sst{\mbox{SST-2}\xspace}
\newcommand\ptft{pre-training$+$fine-tuning\xspace}
\newcommand\Ptft{Pre-training$+$fine-tuning\xspace}
\newcommand{\bb}{BERT\textsubscript{BASE}\xspace}

\setlength{\marginparwidth}{3.2cm}
\newcounter{mw}
\newcommand{\mw}[1]{%
\refstepcounter{mw}%
{%
\todo[color=orange, size=\footnotesize]{%
[\textbf{mw:\themw}] #1}%
}}%

\newcounter{iu}
\newcommand{\iu}[1]{%
\refstepcounter{iu}%
{%
\todo[color=green, size=\footnotesize]{%
[\textbf{iu:\theiu}] #1}%
}}%

\newcounter{kl}
\newcommand{\kl}[1]{%
\refstepcounter{kl}%
{%
\todo[color=magenta, size=\footnotesize]{%
[\textbf{kl:\thekl}] #1}%
}}%

\newcounter{kt}
\newcommand{\kt}[1]{%
\refstepcounter{kt}%
{%
\todo[color=yellow, size=\footnotesize]{%
[\textbf{kt:\thekt}] #1}%
}}%

\newcommand{\recipename}{Pre-trained Distillation\xspace}
\newcommand{\recipeshortname}{PD\xspace}
\newcommand{\BCW}{BookCorpus \& English Wikipedia}

\newcommand{\trainmarker}{x}
\newcommand{\traincolor}{black}

\newcommand{\teachercolor}{red}
\newcommand{\teachermarker}{o*}

% vt = vanilla training
\newcommand{\VT}{Vanilla Training\xspace}
\newcommand{\vtcolor}{black}
\newcommand{\vtmarker}{x}

% kd = knowledge distillation
\newcommand{\kdcolor}{orange}
\newcommand{\kdmarker}{triangle*}

% wp = wordpiece pre-training
\newcommand{\wpcolor}{purple}
\newcommand{\wpmarker}{pentagon*}

\newcommand{\PtFt}{Pre-training$+$Fine-tuning\xspace}
\newcommand{\pretraincolor}{blue}
\newcommand{\pretrainmarker}{diamond*}

% deft = our algorithm
\newcommand{\deftcolor}{black!30!green}
\newcommand{\deftmarker}{square*}

\newcommand{\D}{\mathcal{D}}
\newcommand{\DLM}{$\D_{LM}$\xspace}
\newcommand{\DL}{$\D_L$\xspace}
\newcommand{\DT}{$\D_T$\xspace}
\newcommand{\nlistar}{NLI*\xspace}

\def\checkmark{\tikz\fill[scale=0.4](0,.35) -- (.25,0) -- (1,.7) -- (.25,.15) -- cycle;}


\newcommand{\bertsizestab}{
    \begin{tabular}{rrrrr}
        \toprule
        & \multicolumn{1}{c}{\bfseries H=128}
        & \multicolumn{1}{c}{\bfseries H=256}
        & \multicolumn{1}{c}{\bfseries H=512}
        & \multicolumn{1}{c}{\bfseries H=768}  \\
        \midrule
        \textbf{L=2}  & \underline{4.4} &  9.7 & 22.8 &  39.2    \\
        \textbf{L=4}  & 4.8 & \underline{11.3}  & \underline{29.1} &  53.4    \\
        \textbf{L=6}  & 5.2 & 12.8 & 35.4 &  67.5    \\
        \textbf{L=8}  & 5.6 & 14.4 & \underline{41.7} &  81.7   \\
        \textbf{L=10} & 6.0 & 16.0 & 48.0 &  95.9    \\
        \textbf{L=12} & 6.4 & 17.6 & 54.3 & \underline{110.1}    \\
        \bottomrule
    \end{tabular}
    \caption{Millions of parameters}
    \label{tab:bertsizes}
}


\newcommand{\bertJFlatenciesspeeduptab}{
    \begin{tabular}{rrrrr}
        \toprule
        & \multicolumn{1}{c}{\bfseries H=128}
        & \multicolumn{1}{c}{\bfseries H=256}
        & \multicolumn{1}{c}{\bfseries H=512}
        & \multicolumn{1}{c}{\bfseries H=768}  \\
        \midrule 
        \textbf{L=2}  & \underline{65.24}  &  31.25 & 14.44  &  7.46  \\
        \textbf{L=4}  & 32.37 &  \underline{15.96} &  \underline{7.27}  & 3.75 \\
        \textbf{L=6}  & 21.87 &  10.67 & 4.85 & 2.50 \\
        \textbf{L=8}  & 16.42 &  8.01 & \underline{3.64} & 1.88 \\
        \textbf{L=10} & 13.05 & 6.37 & 2.90 & 1.50 \\
        \textbf{L=12} & 11.02 & 5.35 & 2.43 & \underline{1.25} \\
        \bottomrule
    \end{tabular}
    \caption{Relative speedup wrt \bertlarge on TPU v2}
    \label{tab:bertlatenciesspeedup}
}

% Commands for accuracy plots.
% arg1 = mark shape (e.g. triangle)
% arg2 = mark and line color (e.g. green)
% arg3 = line type (e.g. solid or dashed)
\newcommand{\addbertplot}[3]{
    \addplot+[mark=#1, mark options={#2, solid}, mark size=2.5pt, #2, #3]
        plot [error bars/.cd, y dir = both, y explicit]
        table [x=index, y=mean, col sep=comma] % y error=sem
}

% % No idea why the first two lines under "Label" are ignored... "x" is just a placeholder.
\begin{filecontents}{bert_sizes.dat} 
Label
x
x
{Tiny\\2L/128H} 
{Mini\\4L/256H}
{Small\\4L/512H}
{Medium\\8L/512H}
{Base\\12L/768H}
\end{filecontents}

% Use this with enlargelimits=false
\begin{filecontents}{bert_sizes_v2.dat} 
Label
x
{Tiny\\2L/128H} 
{Mini\\4L/256H}
{Small\\4L/512H}
{Medium\\8L/512H}
{Base\\12L/768H}
\end{filecontents}

\begin{filecontents}{bert_sizes_v3.dat} 
Label
x
x
{Tiny\\2L\\128H} 
{Mini\\4L\\256H}
{Small\\4L\\512H}
{Medium\\8L\\512H}
{Base\\12L\\768H}
\end{filecontents}

\newcommand{\fivepoints}{
    \pgfplotsset{
        every tick label/.append style={font=\scriptsize},
        enlargelimits=true,
        grid=minor,
        legend pos=north west,
        legend cell align={left},
        legend columns=4,
        legend style={font=\tiny},
        xlabel style={font=\scriptsize},
        ylabel style={font=\scriptsize},
        title style={font=\small},
    }
}

\newcommand{\newfivepoints}{
    \pgfplotsset{
        every tick label/.append style={font=\scriptsize},
        enlargelimits=true,
        grid=minor,
        legend pos=north west,
        legend cell align={left},
        legend columns=4,
        legend style={font=\small},
        xlabel style={font=\scriptsize},
        ylabel style={font=\footnotesize},
        title style={font=\small},
    }
}

\newcommand{\pgfaccuracynoscale}{
    \pgfplotsset{
        every tick label/.append style={font=\tiny},
        legend style={font=\tiny},
        xlabel style={font=\tiny},
        ylabel style={font=\tiny},
        enlargelimits=false,
        grid=major,
        xtick=\empty,
        extra x ticks={
             1,  2,  3,  4,  5,  6,
             7,  8,  9, 10, 11, 12,
            13, 14, 15, 16, 17, 18,
            19, 20, 21, 22, 23, 24},
        extra x tick labels={
            2/128, 4/128, 6/128, 8/128, 10/128, 12/128,
            2/256, 4/256, 6/256, 8/256, 10/256, 12/256,
            2/512, 4/512, 6/512, 8/512, 10/512, 12/512,
            2/768, 4/768, 6/768, 8/768, 10/768, 12/768,
        },
        extra x tick style={tick label style={rotate=90}},
        legend pos=north west,
        legend cell align={left},
        legend columns=4,
        legend style={font=\tiny},
    }
}

\newcommand{\pgfaccuracy}{
    \pgfaccuracynoscale
    \pgfplotsset{
        % scale only axis,
        width=\textwidth * 1.2,
        % tweaked so that 3 plots fit vertically on the same page
        height=\paperheight / 4.7,
    }
}

\newcommand{\pgfaccuracybylatency}{
    \pgfaccuracynoscale
    \pgfplotsset{
        extra x tick labels={
             2/128,  4/128,  2/256, 6/128,  8/128,  4/256, 
             2/512, 10/128, 12/128, 6/256,  8/256,  4/512, 
             2/768, 10/256, 12/256, 6/512,  4/768,  8/512, 
            10/512, 6/768,  12/512, 8/768, 10/768, 12/768, 
        },
    }
}

%\usetikzlibrary{shapes,arrows, positioning}

% \newcolumntype{N}{>{\centering\arraybackslash}X}
% \newcolumntype{P}[1]{>{\centering\arraybackslash}p{#1}}
% \newcolumntype{L}[1]{>{\raggedright\arraybackslash}p{#1}}

% \definecolor{g-red}{HTML}{DB4437}
% \definecolor{g-blue}{HTML}{4285F4}
% \definecolor{g-green}{HTML}{0F9D58}
% \definecolor{g-yellow}{HTML}{F4B400}
% \definecolor{g-orange}{HTML}{FF9800}
% \definecolor{g-grey}{HTML}{9E9E9E}

\newcommand\yesmark{\ding{51}}
\renewcommand{\cite}{\citep}

% \tikzstyle{pretrain} = [draw, fill=g-green!20, rectangle, anchor=south, align=center, text width=8em, minimum height=3em]
% \tikzstyle{distill} = [draw, fill=g-blue!20, rectangle, anchor=south, align=center, text width=8em, minimum height=3em]
% \tikzstyle{fine-tune} = [draw, fill=g-red!20, rectangle, anchor=south, align=center, text width=8em, minimum height=3em]
% \tikzstyle{teacher} = [draw, align=center, text width=8em, anchor=north, node distance=0cm, minimum height=1em]
% \tikzstyle{score} = [draw, minimum width=2em]
% \tikzstyle{dep} = [->, looseness=0.4]